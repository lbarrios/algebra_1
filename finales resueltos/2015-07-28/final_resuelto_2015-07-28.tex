%------------------------------------------------------------------------------
% PACKAGES AND OTHER DOCUMENT CONFIGURATIONS
%------------------------------------------------------------------------------
\documentclass[a4paper,spanish]{article}
\usepackage{header}
\begin{document}
%------------------------------------------------------------------------------
% TITLE SECTION
%------------------------------------------------------------------------------
\title{Álgebra I - Final}
\author{Leandro Ezequiel Barrios}
\date{28/07/2015}
\maketitle

%------------------------------------------------------------------------------
% Ejercicio 1
%------------------------------------------------------------------------------
\section{%
  Sea $\mathfrak{R}$ la relación en $A := \{10, ..., 1000\}$ %
  definida por $(n:m) \neq 1.$%
}
  %----------------------------------------------------------------------------
  % Item A
  %----------------------------------------------------------------------------
  \subsection{%
    Determine si $\mathfrak{R}$ es reflexiva, simétrica, antisimétrica %
    o transitiva.%
  }

  \begin{itemize}
    %--------------------------------------------------------------------------
    % Reflexividad
    %--------------------------------------------------------------------------
    \item Reflexividad: $\forall a \in A,\> a\mathfrak{R}a$
    \begin{align*}
      a\mathfrak{R}a \iif (a:a) &\neq 1 \\
      (a:a) &= a \\
      a &\neq 1
    \end{align*}
    Luego, $a\mathfrak{R}a$, entonces $\mathfrak{R}$ es reflexiva.

    %--------------------------------------------------------------------------
    % Simetría
    %--------------------------------------------------------------------------
    \item Simetría: $\forall n,m \in A,\> n\mathfrak{R}m \iif m\mathfrak{R}n$
    \begin{align*}
      n\mathfrak{R}m      &\iif (n:m) \neq 1\\
      (n:m) \neq 1 \quad  &\iif \quad (m:n) \neq 1 \\
      (m:n) \neq 1        &\iif m\mathfrak{R}n
    \end{align*}
    Luego, $n\mathfrak{R}m \iif m\mathfrak{R}n$,
    entonces $\mathfrak{R}$ es simétrica.

    %--------------------------------------------------------------------------
    % Antisimetría
    %--------------------------------------------------------------------------
    \item Antisimetría: $\forall n,m \in A,\> 
    n\mathfrak{R}m \land m\mathfrak{R}n \then n=m$

    Basta con encontrar un contraejemplo para ver que $\mathfrak{R}$ 
    no es antisimétrica.
    \begin{align*}
      (10:20) \neq 1 \iif 10\>\mathfrak{R}\>20 \\
      (20:10) \neq 1 \iif 20\>\mathfrak{R}\>10
    \end{align*}
    $10\mathfrak{R}20 \land 20\mathfrak{R}10$, pero $10 \neq 20$, por lo que
    $\mathfrak{R}$ no es antisimétrica.

    %--------------------------------------------------------------------------
    % Transitividad
    %--------------------------------------------------------------------------
    \item Transitividad: $\forall a,b,c \in A,\>
    a\mathfrak{R}b \land b\mathfrak{R}c \then a\mathfrak{R}c$

    Dados $a,b,c \in A$, $a\mathfrak{R}b$, $b\mathfrak{R}c$, se que
    \begin{align*}
      a\mathfrak{R}b \iif (a:b) \neq 1 \\
      b\mathfrak{R}c \iif (b:c) \neq 1
    \end{align*}

    Basta con encontrar un contraejemplo para ver que $\mathfrak{R}$ 
    no es transitiva.\\
    Dados $a=2,\> b=6,\> c=3$,
    \begin{align*}
      (2:6) = 2 \neq 1 \then 2\>\mathfrak{R}\>6 \\
      (6:3) = 3 \neq 1 \then 6\>\mathfrak{R}\>3 \\
    \end{align*}

    pero $(2:3) = 1$, por lo que $2\> \cancel{\mathfrak{R}}\> 3$.
    Luego, $\mathfrak{R}$ no es transitiva.

  \end{itemize}

  \subsection{%
  Encuentre la cantidad de $m \in A$ tales que $m\> \mathfrak{R}\> 12$%
  }

    Quiero encontrar todos los m tales que $m\>\mathfrak{R}\>12$, siendo que  
    $m\>\mathfrak{R}\>12 \iif (m:12) \neq 1$.

    Factorizando $12$ en primos, se que 
    \[
      12 = 2^2 \cdot 3
    \]

    Entonces, dado $d$ el $mcm$ entre $m$ y $12$, se cumple que
    \[
      d = (m:12) \neq 1
      \iif
      (d \neq 1 \land d \mid m \land d \mid 12)
    \] 

    En particular, 
    \[
      (d \mid 12 \land 12=2^2 \cdot 3)
      \then d \mid 2^2 \cdot 3 
      \then (2 \mid d \lor 3 \mid d)
    \]

    Es decir que $d$ es múltiplo de $2$ o de $3$. Luego, por transitividad
    \begin{align*}
      (d \mid m) \land (2 &\mid d \lor 3 \mid d) \\
      (2\mid d \land d \mid m) &\lor (3 \mid d \land d \mid m) \\
      (2 \mid m)               &\lor (3 \mid m)
    \end{align*}

    Lo que singnifica que basta con ver cuántos $m \in A$ cumplen esta 
    última condición.

    \begin{itemize}

      \item Múltiplos de 2:

      Si tomo el conjunto $A$, y para cada número calculo el resto módulo 2, voy 
      a formar un nuevo conjunto
      \[
        A_2 = \{0, 1, 0, \cdots, 1, 0\}
      \]
      en donde, sin contar el caso del último número, cada dos números 
      uno de ellos es divisible por 2. 

      Luego, dado que A tiene 990 números (sin contar el último),
      la cantidad de números divisibles por 2 es $990/2 + 1 = 496$.

      \item Múltiplos de 3:

      Usando un razonamiento similar al descripto arriba, formo el conjunto
      \[
        A_3 = \{\underbrace{1, 2}_{2}, \underbrace{0, 1, 2, 0, 
        \cdots, 0, 1, 2}_{987}, \underbrace{0, 1}_{2}\}
      \]
      en donde, sin contar los primeros y últimos dos números, cada 3 de ellos
      uno es divisible por 3. Luego, tengo $987/3 + 1 = 990/3 = 330$ números
      divisibles por 3.

      \item Múltiplos de 6:

      Ahora bien, si yo sumo la cantidad de múltiplos de 2 más la cantidad de
      múltiplos de 3, estoy contando 2 veces a los números que son múltiplos de 6,
      que son múltiplos de 2 y de 3. Para evitar esto, voy a calcular la cantidad
      de múltiplos de 2 más la cantidad de múltiplos de 3 menos la cantidad de
      múltiplos de 6.

      Usando el mismo razonamiento, formo el conjunto
      \[
        B_6 = \{\underbrace{4, 5}_{2}, \underbrace{0, 1, 2, 3, 4, 5, 
        \cdots, 0, 1, 2, 3, 4, 5}_{984}, \underbrace{0, 1, 2, 3, 4}_{5} \}
      \]
      en donde, sin contar los primeros 2 y los últimos 5, tengo 984 números
      en donde cada 6 números, uno de ellos es múltiplo de 6.

      Luego, tengo $984/6 + 1 = 990/6 = 165$ múltiplos de 6.

    \end{itemize}

    Finalmente, tengo $496 + 330 - 165 = 661$ múltiplos de 2 o de 3 que 
    pertenecen a $A$. 

    Es decir que tengo $661$ posibles $m$ tales que
    $m\>\mathfrak{R}\>12$.

%------------------------------------------------------------------------------
% Ejercicio 2
%------------------------------------------------------------------------------
\section{%
  Sea $a \in \Z$ tal que $(a^{182}-26:130) = 13$. %
  Calcule $(a^{25}-39:2 \cdot 5^3 \cdot 13^2)$%
}

\newcommand{\equivfive}{\underset{(5)}\equiv}
\newcommand{\nequivfive}{\underset{(5)}\equiv}

  \begin{align*}
    (a^{182}-26:130) = 13 \then 13   &\mid a^{182}-26                       \\
                            \iif 13   &\mid a^{182} - 26 + (13 \cdot 2)     \\
                            \iif 13   &\mid a^{182}                         \\
                            \iif 13   &\mid a                               \\
    \\
    2 \mid 130 \land 2 \nmid (a^{182}-26:130) %
                          \then 2     &\nmid a^{182}-26                     \\
                            \iif 2    &\nmid a^{182} - 26 + (2 \cdot 13)    \\
                            \iif 2    &\nmid a^{182}                        \\
                            \iif 2    &\nmid a                              \\
    \\
                            13 \mid a \iif 13^{25} &\mid a^{25}             \\
    13^2 \mid 13^{25} \land 13^{25} \mid a^{25} %
                           \then 13^2 &\mid a^{25}                          \\
    13 \mid 39 \land 13^2 \mid a^{25} %
                           \then 13^2 &\mid a^{25} + 39                     \\
    \\
                    2 \nmid a \then 2 &\nmid a^25                           \\
    2 \nmid 39 \land 2 \nmid a^{25} %
                              \then 2 &\mid a^{25} - 39                     \\
    \\
    5 \mid 130 \land 5 \nmid (a^{182}-26:130) %
                              \then 5 &\nmid a^{182} - 26
  \end{align*}

  \subsection*{Caso $a \equivfive 0$:}
  \begin{align*}
    a \equivfive 0 \then a^{182} - 26 &\equivfive 0^{182} - 26              \\
                   \iif  a^{182} - 26 &\equivfive -26                       \\
                   \iif  a^{182} - 26 &\equivfive -1                        \\
                   \iif  a^{182} - 26 &\equivfive 4                         \\
    a \equivfive 0 \then a^{25} - 39  &\equivfive 0^{25} - 39               \\
                   \iif  a^{25} - 39  &\equivfive -39                       \\
                   \iif  a^{25} - 39  &\equivfive -4                        \\
                   \iif  a^{25} - 39  &\equivfive 1
  \end{align*}

  es decir que, para $a \equivfive 0$, $a^{25} - 39$ no es múltiplo de 5.

  \subsection*{Caso $a \equivfive 1$:}
  \begin{align*}
    a \equivfive 1 \then a^{182} - 26 &\equivfive 1^{182} - 26              \\
                   \iif  a^{182} - 26 &\equivfive 1       - 1               \\
                   \iif  a^{182} - 26 &\equivfive 0
  \end{align*}

  pero $5 \nmid a^{182} - 26$ por lo que $a \not\equivfive 1$.

  \subsection*{Caso $a \equivfive 2$:}
  \begin{align*}
    a \equivfive 2 \then a^{182} - 26 &\equivfive 2^{182} - 26              \\
                   \iif  a^{182} - 26 &\equivfive (2^2)^{91}  - 1           \\
                   \iif  a^{182} - 26 &\equivfive 4^{90 + 1}  - 1           \\
                   \iif  a^{182} - 26 &\equivfive (4^2)^{45} \cdot 4^1  - 1 \\
                   \iif  a^{182} - 26 &\equivfive 16^{45} \cdot 4  - 1      \\
                   \iif  a^{182} - 26 &\equivfive 1^{45} \cdot 4  - 1       \\
                   \iif  a^{182} - 26 &\equivfive 3                         \\
    a \equivfive 2 \then a^{25} - 39  &\equivfive 2^{25} - 39               \\
                   \iif  a^{25} - 39  &\equivfive 2^{24+1} + 1              \\
                   \iif  a^{25} - 39  &\equivfive (2^2)^{12} \cdot 2 + 1    \\
                   \iif  a^{25} - 39  &\equivfive 4^{12} \cdot 2 + 1        \\
                   \iif  a^{25} - 39  &\equivfive (4^2)^{6} \cdot 2 + 1     \\
                   \iif  a^{25} - 39  &\equivfive 16^{6} \cdot 2 + 1        \\
                   \iif  a^{25} - 39  &\equivfive 1^{6} \cdot 2 + 1         \\
                   \iif  a^{25} - 39  &\equivfive 2 + 1                     \\
                   \iif  a^{25} - 39  &\equivfive 3
  \end{align*}

  es decir que, para $a \equivfive 2$, $a^{25} - 39$ no es múltiplo de 5.

  \subsection*{Caso $a \equivfive 3$:}
  \begin{align*}
    a \equivfive 2 \then a^{182} - 26 &\equivfive 3^{182} - 26              \\
                   \iif  a^{182} - 26 &\equivfive (3^2)^{91} - 1            \\
                   \iif  a^{182} - 26 &\equivfive 9^{91} - 1                \\
                   \iif  a^{182} - 26 &\equivfive 4^{90 + 1} - 1            \\
                   \iif  a^{182} - 26 &\equivfive (4^2)^{45} \cdot 4^1 - 1  \\
                   \iif  a^{182} - 26 &\equivfive 16^{45} \cdot 4 - 1       \\
                   \iif  a^{182} - 26 &\equivfive 1^{45} \cdot 4 - 1        \\
                   \iif  a^{182} - 26 &\equivfive 4 - 1                     \\
                   \iif  a^{182} - 26 &\equivfive 3                         \\
    a \equivfive 2 \then a^{25} - 39  &\equivfive 3^{25} - 39               \\
                   \iif  a^{25} - 39  &\equivfive 3^{24 + 1} + 1            \\
                   \iif  a^{25} - 39  &\equivfive (3^2)^{12} \cdot 3 + 1    \\
                   \iif  a^{25} - 39  &\equivfive 9^{12} \cdot 3 + 1        \\
                   \iif  a^{25} - 39  &\equivfive 4^{12} \cdot 3 + 1        \\
                   \iif  a^{25} - 39  &\equivfive (4^2)^{6} \cdot 3 + 1     \\
                   \iif  a^{25} - 39  &\equivfive (16)^{6} \cdot 3 + 1      \\
                   \iif  a^{25} - 39  &\equivfive (1)^{6} \cdot 3 + 1       \\
                   \iif  a^{25} - 39  &\equivfive 3 + 1                     \\
                   \iif  a^{25} - 39  &\equivfive 4                         \\
  \end{align*}

  es decir que, para $a \equivfive 3$, $a^{25} - 39$ no es múltiplo de 5.

  \subsection*{Caso $a \equivfive 4$:}
  \begin{align*}
    a \equivfive 2 \then a^{182} - 26 &\equivfive 4^{182} - 26              \\
                   \iif  a^{182} - 26 &\equivfive (4^2)^{91} - 1            \\
                   \iif  a^{182} - 26 &\equivfive 16^{91} - 1               \\
                   \iif  a^{182} - 26 &\equivfive 1^{91} - 1                \\
                   \iif  a^{182} - 26 &\equivfive 0                         \\
  \end{align*}

  pero $5 \nmid a^{182} - 26$ por lo que $a \not\equivfive 4$.

  \subsection*{}
  Entonces, cualquiera sea el a, siempre que cumpla $(a^{182}-26:130) = 13$,
  se que $5 \nmid a^{182} - 26$. 

  Finalmente, dadas las siguientes condiciones, 

  \begin{align*}
    2 \mid 2 \cdot 5^3 \cdot 13^2     &\land 2 \mid a^{25}-39 \\
    5 \mid 2 \cdot 5^3 \cdot 13^2     &\land 5 \nmid a^{25} - 39 \\ 
    13 \mid 2 \cdot 5^3 \cdot 13^2    &\land 13 \mid a^{25} - 39 \\
    13^2 \mid 2 \cdot 5^3 \cdot 13^2  &\land 13^2 \nmid a^{25} - 39 \\
  \end{align*}
  es posible concluir que
  \begin{empheq}[box=\widefbox]{align*}
  (a^{25}-39 : 2 \cdot 5^3 \cdot 13^2) = 2 \cdot 13 = 26
  \end{empheq}

%------------------------------------------------------------------------------
% Ejercicio 3
%------------------------------------------------------------------------------
\section{%
  Para cada $n \in \N$, encuentre el resto de dividir por $7$ a $n^{3n}$%
  en términos de una congruencia apropiada de $n$.%
}

\newcommand{\equivseven}{\underset{(7)}\equiv}
\newcommand{\equivtwo}{\underset{(2)}\equiv}
\begin{itemize}

%%%----------------------------------------------------------------------------
%%% Caso n==0 (7)
%%%----------------------------------------------------------------------------
\item
Supongo $n \equivseven 0 $,
\[
  n^{3n} \equivseven 0^{3n} \equivseven 0 
\]\[ \Leftrightarrow \]
\begin{center}\fbox{$
  n^{3n} \equivseven 0
$}\end{center}

%%%----------------------------------------------------------------------------
%%% Caso n==1 (7)
%%%----------------------------------------------------------------------------
\item
Supongo $n \equivseven 1$,
\[
  n^{3n} \equivseven 1^{3n} \equivseven 1 
\]\[ \Leftrightarrow \]
\begin{center}\fbox{$
  n^{3n} \equivseven 1 
$}\end{center}

%%%----------------------------------------------------------------------------
%%% Caso n==2 (7)
%%%----------------------------------------------------------------------------
\item
Supongo $n \equivseven 2$,
\[
  n^{3n} \equivseven 2^{3n} \equivseven {(2^3)}^n \equivseven 8^n (7) \equivseven 1^n \equivseven 1
\]\[ \Leftrightarrow \]
\begin{center}\fbox{$
  n^{3n} \equivseven 1
$}\end{center}

%%%----------------------------------------------------------------------------
%%% Caso n==3 (7)
%%%----------------------------------------------------------------------------
\item
Supongo $n \equivseven 3$,

\[
  n^{3n} \equivseven 3^{3n} \equivseven {(3^3)}^n \equivseven 27^n \equivseven 6^n
\]\[ \Leftrightarrow \]\[
  n^{3n} \equivseven 6^n
\]

...además,
\[
  n \equivseven 3 \then \exists k \in \Z: n = 7k + 3
\]

...entonces, para algún $k \in \Z$,
\begin{align*}
n^{3n} &\equivseven 6^n = 6^{7k+3}                    \\
n^{3n} &\equivseven {(6^7)}^k * 6^3                   \\
n^{3n} &\equivseven {(6 * {(6^2)}^3)}^k * 6 * 6^2     \\
n^{3n} &\equivseven 6^k * {(36^3)}^k * 6 * 36         \\
n^{3n} &\equivseven 6^k * {(1^3)}^k * 6 * 1           \\
n^{3n} &\equivseven 6^k * 1^k * 6                     \\
n^{3n} &\equivseven 6^k * 6                           \\
n^{3n} &\equivseven 6^{k+1}                           \\
\end{align*}

\[
\left\{
\begin{array}{lrl}
Caso:~ k \equivtwo 0 &\iif k+1 \equivtwo 1 & \iif n = 7k + 3 \equivtwo 3 \equivtwo 1 \\
&\then n^{3n} &\equivseven 6^{k+1} \\
& \cdots &\equivseven 6^{k} * 6 \\
& \cdots &\equivseven {(6^2)}^{k/2} * 6 \\
& \cdots &\equivseven 1^k * 6 \\
& \cdots &\equivseven 6 \\
\\
Caso:~ k \equivtwo 1 &\iif k+1 \equivtwo 0 & \iif n = 7k + 3 \equivtwo 7+3 \equivtwo 0 \\
&\then n^{3n} &\equivseven 6^{k+1} \\
& \cdots &\equivseven {(6^2)}^{(k+1)/2} \\
& \cdots &\equivseven 1^{(k+1)/2} \\
& \cdots &\equivseven 1 \\
\end{array}
\right .
\]

\begin{empheq}[box=\widefbox]{align*}
  n \equivtwo 0 &\then n^{3n} \equivseven 1\\
  n \equivtwo 1 &\then n^{3n} \equivseven 6
\end{empheq}

%%%----------------------------------------------------------------------------
%%% Caso n==4 (7)
%%%----------------------------------------------------------------------------
\item
Supongo $n \equiv 4 (7)$,

\[
  n^{3n} \equiv 4^{3n} \equiv {(4^3)}^n \equiv 64^n \equiv 1^n \equiv 1 (7)
\]\[ \Leftrightarrow \]
\begin{center}\fbox{$
  n^{3n} \equiv 1 (7)
$}\end{center}

%%%----------------------------------------------------------------------------
%%% Caso n==5 (7)
%%%----------------------------------------------------------------------------
\item
Supongo $n \equivseven 5$,

\[
  n^{3n} \equivseven 5^{3n} \equivseven {(5^3)}^n \equivseven 125^n \equivseven 6^n
\]\[ \Leftrightarrow \]\[
  n^{3n} \equivseven 6^n
\]

...además,
\[
  n \equivseven 5 \then \exists k \in \Z: n = 7k + 5
\]

...entonces, para algún $k \in \Z$,
\begin{align*}
n^{3n} &\equivseven 6^n \\
n^{3n} &\equivseven 6^{7k+5} \\
n^{3n} &\equivseven {(6^7)}^k * 6^5 \\
n^{3n} &\equivseven {(6 * {(6^2)}^3)}^k * 6 * {(6^2)}^2 \\
n^{3n} &\equivseven 6^k * {(36^3)}^k * 6 * 36^2 \\
n^{3n} &\equivseven 6^k * {(1^3)}^k * 6 * 1^2 \\
n^{3n} &\equivseven 6^k * 1^k * 6 \\
n^{3n} &\equivseven 6^k * 6 \\
n^{3n} &\equivseven 6^{k+1}
\end{align*}

\[
\left\{
\begin{array}{lrl}
Caso:~ k \equivtwo 0 &\iif k+1 \equivtwo 1 & \iif n = 7k + 5 \equivtwo 5 \equivtwo 1 \\
&\then n^{3n} &\equivseven 6 \\
\\
Caso:~ k \equivtwo 1 &\iif k+1 \equivtwo 0 & \iif n = 7k + 5 \equivtwo 7+5 \equivtwo 0 \\
&\then n^{3n} &\equivseven 1 \\
\end{array}
\right.
\]

\begin{empheq}[box=\widefbox]{align*}
  n \equivtwo 0 &\then n^{3n} \equivseven 1\\
  n \equivtwo 1 &\then n^{3n} \equivseven 6
\end{empheq}

%%%----------------------------------------------------------------------------
%%% Caso n==6 (7)
%%%----------------------------------------------------------------------------
\item
Supongo $n \equiv 6 (7)$,

\[
  n^{3n} \equiv 6^{3n} \equiv {(6^3)}^n \equiv 216^n \equiv 6^n (7)
\]\[ \Leftrightarrow \]\[
  n^{3n} \equiv 1 (7)
\]

...además,
\[
  n \equivseven 6 \then \exists k \in \Z: n = 7k + 6
\]

...entonces, para algún $k \in \Z$,
\begin{align*}
n^{3n} &\equivseven 6^n \\
n^{3n} &\equivseven 6^{7k+6} \\
n^{3n} &\equivseven {(6^7)}^k * 6^6 \\
n^{3n} &\equivseven {(6 * {(6^2)}^3)}^k * {(6^2)}^3 \\
n^{3n} &\equivseven 6^k * {(36^3)}^k * 36^3 \\
n^{3n} &\equivseven 6^k * {(1^3)}^k * 1^3 \\
n^{3n} &\equivseven 6^k * 1^k \\
n^{3n} &\equivseven 6^k
\end{align*}

\[
\left\{
\begin{array}{lrl}
Caso:~ k \equivtwo 0 &\iif k+1 \equivtwo 1 & \iif n = 7k + 6 \equivtwo 6 \equivtwo 0 \\
                                &                   \then n^{3n} & \equivseven 6^k \\
                                &                         \cdots & \equivseven {(6^2)}^{k/2} \\
                                &                         \cdots & \equivseven 36^{k/2} \\
                                &                         \cdots & \equivseven 1^{k/2} \\
                                &                         \cdots & \equivseven 1 \\
Caso:~ k \equivtwo 1 &\iif k+1 \equivtwo 0 & \iif n = 7k + 6 \equivtwo 7+6 \equivtwo 1 \\
                                &                   \then n^{3n} & \equivseven 6^k \\
                                &                         \cdots & \equivseven 6^{k-1+1} \\
                                &                         \cdots & \equivseven 6^{k-1} * 6 \\
                                &                         \cdots & \equivseven 6^{2*(k-1)/2} * 6 \\
                                &                         \cdots & \equivseven {(6^2)}^{(k-1)/2} * 6 \\
                                &                         \cdots & \equivseven {36}^{(k-1)/2} * 6 \\
                                &                         \cdots & \equivseven {1}^{(k-1)/2} * 6 \\
                                &                         \cdots & \equivseven 1^{k/2} *6 \\
                                &                         \cdots & \equivseven 6 \\
\end{array}
\right.
\]

\begin{empheq}[box=\widefbox]{align*}
  n \equivtwo 0 &\then n^{3n} \equivseven 1\\
  n \equivtwo 1 &\then n^{3n} \equivseven 6
\end{empheq}

\end{itemize}

%------------------------------------------------------------------------------
% Ejercicio 3
%------------------------------------------------------------------------------
\section{%
Sea $f \in \Q[X]$ el polinomio $f = X^4 + X^3 + X^2 + X + 1$.%
}
  %------------------------------------------------------------------------------
  % Item A
  %------------------------------------------------------------------------------
  \subsection{%
    Pruebe que $f$ es irreductible en $\Q[X]$.%
  }

  %------------------------------------------------------------------------------
  % Item B
  %------------------------------------------------------------------------------
  \subsection{%
    Para cada número natural $n$ calcule $(f:X^n-1)$.%
  }
  
  %------------------------------------------------------------------------------
  % Item C
  %------------------------------------------------------------------------------
  \subsection{%
    Pruebe que si $p \in \Q[X]$ tiene como raíz a alguna raíz quinta %
    primitiva de la unidad, entonces todas las raíces quintas primitivas %
    de la unidad son raíces de $p$%
  }

\end{document}