%------------------------------------------------------------------------------
% PACKAGES AND OTHER DOCUMENT CONFIGURATIONS
%------------------------------------------------------------------------------
\documentclass[a4paper,spanish]{article}
\usepackage{header}
\begin{document}
%------------------------------------------------------------------------------
% TITLE SECTION
%------------------------------------------------------------------------------
\title{Álgebra I - Final}
\author{Leandro Ezequiel Barrios}
\date{28/07/2015}
\maketitle

%------------------------------------------------------------------------------
% Ejercicio 1
%------------------------------------------------------------------------------
\section{%
  Sea $\mathfrak{R}$ la relación en $A := \{10, ..., 1000\}$ %
  definida por $(n:m) \neq 1.$%
}

\subsection{Determine si $\mathfrak{R}$ es reflexiva, simétrica, antisimétrica o transitiva.}
\subsection{Encuentre la cantidad de $m \in A$ tales que $m \mathfrak{R} 12$}

%------------------------------------------------------------------------------
% Ejercicio 2
%------------------------------------------------------------------------------
\section{%
  Sea $a \in \Z$ tal que $(a^{182}-26:130) = 13$. %
  Calcule $(a^{25}-39:2 \cdot 5^3 \cdot 13^2)$%
}

%------------------------------------------------------------------------------
% Ejercicio 3
%------------------------------------------------------------------------------
\section{%
  Para cada $n \in \N$, encuentre el resto de dividir por $7$ a $n^{3n}$%
  en términos de una congruencia apropiada de $n$.%
}

\newcommand{\equivseven}{\underset{(7)}\equiv}
\newcommand{\equivtwo}{\underset{(2)}\equiv}
\begin{itemize}

%%%----------------------------------------------------------------------------
%%% Caso n==0 (7)
%%%----------------------------------------------------------------------------
\item
Supongo $n \equivseven 0 $,
\[
  n^{3n} \equivseven 0^{3n} \equivseven 0 
\]\[ \Leftrightarrow \]
\begin{center}\fbox{$
  n^{3n} \equivseven 0
$}\end{center}

%%%----------------------------------------------------------------------------
%%% Caso n==1 (7)
%%%----------------------------------------------------------------------------
\item
Supongo $n \equivseven 1$,
\[
  n^{3n} \equivseven 1^{3n} \equivseven 1 
\]\[ \Leftrightarrow \]
\begin{center}\fbox{$
  n^{3n} \equivseven 1 
$}\end{center}

%%%----------------------------------------------------------------------------
%%% Caso n==2 (7)
%%%----------------------------------------------------------------------------
\item
Supongo $n \equivseven 2$,
\[
  n^{3n} \equivseven 2^{3n} \equivseven {(2^3)}^n \equivseven 8^n (7) \equivseven 1^n \equivseven 1
\]\[ \Leftrightarrow \]
\begin{center}\fbox{$
  n^{3n} \equivseven 1
$}\end{center}

%%%----------------------------------------------------------------------------
%%% Caso n==3 (7)
%%%----------------------------------------------------------------------------
\item
Supongo $n \equivseven 3$,

\[
  n^{3n} \equivseven 3^{3n} \equivseven {(3^3)}^n \equivseven 27^n \equivseven 6^n
\]\[ \Leftrightarrow \]\[
  n^{3n} \equivseven 6^n
\]

...además,
\[
  n \equivseven 3 \then \exists k \in \Z: n = 7k + 3
\]

...entonces, para algún $k \in \Z$,
\begin{align*}
n^{3n} &\equivseven 6^n = 6^{7k+3}                    \\
n^{3n} &\equivseven {(6^7)}^k * 6^3                   \\
n^{3n} &\equivseven {(6 * {(6^2)}^3)}^k * 6 * 6^2     \\
n^{3n} &\equivseven 6^k * {(36^3)}^k * 6 * 36         \\
n^{3n} &\equivseven 6^k * {(1^3)}^k * 6 * 1           \\
n^{3n} &\equivseven 6^k * 1^k * 6                     \\
n^{3n} &\equivseven 6^k * 6                           \\
n^{3n} &\equivseven 6^{k+1}                           \\
\end{align*}

\[
\left\{
\begin{array}{lrl}
Caso:~ k \equivtwo 0 &\iif k+1 \equivtwo 1 & \iif n = 7k + 3 \equivtwo 3 \equivtwo 1 \\
&\then n^{3n} &\equivseven 6^{k+1} \\
& \cdots &\equivseven 6^{k} * 6 \\
& \cdots &\equivseven {(6^2)}^{k/2} * 6 \\
& \cdots &\equivseven 1^k * 6 \\
& \cdots &\equivseven 6 \\
\\
Caso:~ k \equivtwo 1 &\iif k+1 \equivtwo 0 & \iif n = 7k + 3 \equivtwo 7+3 \equivtwo 0 \\
&\then n^{3n} &\equivseven 6^{k+1} \\
& \cdots &\equivseven {(6^2)}^{(k+1)/2} \\
& \cdots &\equivseven 1^{(k+1)/2} \\
& \cdots &\equivseven 1 \\
\end{array}
\right .
\]

\begin{empheq}[box=\widefbox]{align*}
  n \equivtwo 0 &\then n^{3n} \equivseven 1\\
  n \equivtwo 1 &\then n^{3n} \equivseven 6
\end{empheq}

%%%----------------------------------------------------------------------------
%%% Caso n==4 (7)
%%%----------------------------------------------------------------------------
\item
Supongo $n \equiv 4 (7)$,

\[
  n^{3n} \equiv 4^{3n} \equiv {(4^3)}^n \equiv 64^n \equiv 1^n \equiv 1 (7)
\]\[ \Leftrightarrow \]
\begin{center}\fbox{$
  n^{3n} \equiv 1 (7)
$}\end{center}

%%%----------------------------------------------------------------------------
%%% Caso n==5 (7)
%%%----------------------------------------------------------------------------
\item
Supongo $n \equivseven 5$,

\[
  n^{3n} \equivseven 5^{3n} \equivseven {(5^3)}^n \equivseven 125^n \equivseven 6^n
\]\[ \Leftrightarrow \]\[
  n^{3n} \equivseven 6^n
\]

...además,
\[
  n \equivseven 5 \then \exists k \in \Z: n = 7k + 5
\]

...entonces, para algún $k \in \Z$,
\begin{align*}
n^{3n} &\equivseven 6^n \\
n^{3n} &\equivseven 6^{7k+5} \\
n^{3n} &\equivseven {(6^7)}^k * 6^5 \\
n^{3n} &\equivseven {(6 * {(6^2)}^3)}^k * 6 * {(6^2)}^2 \\
n^{3n} &\equivseven 6^k * {(36^3)}^k * 6 * 36^2 \\
n^{3n} &\equivseven 6^k * {(1^3)}^k * 6 * 1^2 \\
n^{3n} &\equivseven 6^k * 1^k * 6 \\
n^{3n} &\equivseven 6^k * 6 \\
n^{3n} &\equivseven 6^{k+1}
\end{align*}

\[
\left\{
\begin{array}{lrl}
Caso:~ k \equivtwo 0 &\iif k+1 \equivtwo 1 & \iif n = 7k + 5 \equivtwo 5 \equivtwo 1 \\
&\then n^{3n} &\equivseven 6 \\
\\
Caso:~ k \equivtwo 1 &\iif k+1 \equivtwo 0 & \iif n = 7k + 5 \equivtwo 7+5 \equivtwo 0 \\
&\then n^{3n} &\equivseven 1 \\
\end{array}
\right.
\]

\begin{empheq}[box=\widefbox]{align*}
  n \equivtwo 0 &\then n^{3n} \equivseven 1\\
  n \equivtwo 1 &\then n^{3n} \equivseven 6
\end{empheq}

%%%----------------------------------------------------------------------------
%%% Caso n==6 (7)
%%%----------------------------------------------------------------------------
\item
Supongo $n \equiv 6 (7)$,

\[
  n^{3n} \equiv 6^{3n} \equiv {(6^3)}^n \equiv 216^n \equiv 6^n (7)
\]\[ \Leftrightarrow \]\[
  n^{3n} \equiv 1 (7)
\]

...además,
\[
  n \equivseven 6 \then \exists k \in \Z: n = 7k + 6
\]

...entonces, para algún $k \in \Z$,
\begin{align*}
n^{3n} &\equivseven 6^n \\
n^{3n} &\equivseven 6^{7k+6} \\
n^{3n} &\equivseven {(6^7)}^k * 6^6 \\
n^{3n} &\equivseven {(6 * {(6^2)}^3)}^k * {(6^2)}^3 \\
n^{3n} &\equivseven 6^k * {(36^3)}^k * 36^3 \\
n^{3n} &\equivseven 6^k * {(1^3)}^k * 1^3 \\
n^{3n} &\equivseven 6^k * 1^k \\
n^{3n} &\equivseven 6^k
\end{align*}

\[
\left\{
\begin{array}{lrl}
Caso:~ k \equivtwo 0 &\iif k+1 \equivtwo 1 & \iif n = 7k + 6 \equivtwo 6 \equivtwo 0 \\
                                &                   \then n^{3n} & \equivseven 6^k \\
                                &                         \cdots & \equivseven {(6^2)}^{k/2} \\
                                &                         \cdots & \equivseven 36^{k/2} \\
                                &                         \cdots & \equivseven 1^{k/2} \\
                                &                         \cdots & \equivseven 1 \\
Caso:~ k \equivtwo 1 &\iif k+1 \equivtwo 0 & \iif n = 7k + 6 \equivtwo 7+6 \equivtwo 1 \\
                                &                   \then n^{3n} & \equivseven 6^k \\
                                &                         \cdots & \equivseven 6^{k-1+1} \\
                                &                         \cdots & \equivseven 6^{k-1} * 6 \\
                                &                         \cdots & \equivseven 6^{2*(k-1)/2} * 6 \\
                                &                         \cdots & \equivseven {(6^2)}^{(k-1)/2} * 6 \\
                                &                         \cdots & \equivseven {36}^{(k-1)/2} * 6 \\
                                &                         \cdots & \equivseven {1}^{(k-1)/2} * 6 \\
                                &                         \cdots & \equivseven 1^{k/2} *6 \\
                                &                         \cdots & \equivseven 6 \\
\end{array}
\right.
\]

\begin{empheq}[box=\widefbox]{align*}
  n \equivtwo 0 &\then n^{3n} \equivseven 1\\
  n \equivtwo 1 &\then n^{3n} \equivseven 6
\end{empheq}

\end{itemize}

%------------------------------------------------------------------------------
% Ejercicio 3
%------------------------------------------------------------------------------
\section{%
Sea $f \in \Q[X]$ el polinomio $f = X^4 + X^3 + X^2 + X + 1$.%
}
  %------------------------------------------------------------------------------
  % Item A
  %------------------------------------------------------------------------------
  \subsection{%
    Pruebe que $f$ es irreductible en $\Q[X]$.%
  }

  %------------------------------------------------------------------------------
  % Item B
  %------------------------------------------------------------------------------
  \subsection{%
    Para cada número natural $n$ calcule $(f:X^n-1)$.%
  }
  
  %------------------------------------------------------------------------------
  % Item C
  %------------------------------------------------------------------------------
  \subsection{%
    Pruebe que si $p \in \Q[X]$ tiene como raíz a alguna raíz quinta %
    primitiva de la unidad, entonces todas las raíces quintas primitivas %
    de la unidad son raíces de $p$%
  }

\end{document}