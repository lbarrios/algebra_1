%------------------------------------------------------------------------------
% PACKAGES AND OTHER DOCUMENT CONFIGURATIONS
%------------------------------------------------------------------------------
\documentclass[a4paper,spanish]{article}
\usepackage{header}
\begin{document}
%------------------------------------------------------------------------------
% TITLE SECTION
%------------------------------------------------------------------------------
\title{Álgebra I - Examen Final\\21/07/2015}
\author{\small Leandro Ezequiel Barrios \\ \footnotesize(lbarrios at dc.uba.ar)}
\date{}
\maketitle

%------------------------------------------------------------------------------
% Ejercicio 1
%------------------------------------------------------------------------------
\newpage
\section{%
  Sean $f(x),g(x) : R \rightarrow R$ \\
  y sean $(f(x))^3 + g(f(x)) \cdot f(x)$ y $f(x^3+g(x) \cdot x)$ biyectivas, \\
  demostrar que $f(x)$ es biyectiva.
}

%------------------------------------------------------------------------------
% Ejercicio 2
%------------------------------------------------------------------------------
\newpage
\section{%
  Sea $p$ primo, demostrar:
}
  %----------------------------------------------------------------------------
  % Item A
  %----------------------------------------------------------------------------
  \subsection{%
    La suma de las raíces primitivas de la unidad $G_p$ es igual a -1
  }

  %----------------------------------------------------------------------------
  % Item A
  %----------------------------------------------------------------------------
  \subsection{%
    La suma de las raíces primitivas de la unidad $G_{p^2}$ es igual a 0
  }

%------------------------------------------------------------------------------
% Ejercicio 3
%------------------------------------------------------------------------------
\newpage
\section{%
  Para qué $p$ primos se cumple que: 
  \[
    2p / 255p+1 + 205
  \]
}

%------------------------------------------------------------------------------
% Ejercicio 4
%------------------------------------------------------------------------------
\newpage
\section{%
  Sean $a,b,c,d \in \N$, demostrar que $X^3+X^2+X+1$ divide a \\
  $X^{4a}+ X^{4b+9} +X^{4c+7}+X^{4d+2}$ en $\Q[x]$
}


\end{document}